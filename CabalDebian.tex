%%
%%
%%

\documentclass[cjk,dvipdfm,12pt]{beamer}
\usetheme{Hibi}
\usefonttheme{professionalfonts}
\usepackage{packages}

\title{Cabalチュートリアル}
%% \subtitle{ -- Debian のすすめ -- }
\author{日比野 啓}
\date{ \today }

\begin{document}

\begin{frame}

\frametitle{Agenda}
\begin{itemize}
\item Cabal
\item Cabal 問題点
\item Debian
\end{itemize}

\end{frame}

\begin{frame}

\frametitle{Agenda}

\begin{itemize}
\item Cabal
\end{itemize}

\end{frame}

\begin{frame}
\frametitle{Cabal - Haskell のパッケージ}

Hackage - Haskell のパッケージ

\begin{itemize}
\item http://hackage.haskell.org/packages/archive/pkg-list.html
\item http://hackage.haskell.org/package/$PACKAGE\_NAME$
\end{itemize}

\end{frame}

\begin{frame}[fragile]
\frametitle{Cabal - install}

%% \lstset{language=bash,basicstyle=\small\ttfamily}
\lstset{language=bash,basicstyle=\ttfamily}

\begin{lstlisting}
% cabal install PACKAGE_NAME
\end{lstlisting}

またはソースを展開した場所で

\begin{lstlisting}
% cabal install
\end{lstlisting}

必要となるパッケージが全てインストールされる

\end{frame}

%% \begin{frame}
%% \frametitle{Cabal - install}
%% \end{frame}

\begin{frame}
\frametitle{Cabal - 難しい依存解決}
%% 問題点その1
\end{frame}

\begin{frame}
\frametitle{Cabal install --max-backjumps=nnn}
\end{frame}


\begin{frame}
\frametitle{Cabal - 壊れる依存関係}
%% 問題点その2
\end{frame}

\begin{frame}
\frametitle{Cabal - バージョンを指定する}
\end{frame}

\begin{frame}
\frametitle{}
\end{frame}

\begin{frame}
\frametitle{Cabal - configure build copy register }
\end{frame}


\begin{frame}
\frametitle{}
\end{frame}

\begin{frame}
\frametitle{Debian}
\end{frame}

\begin{frame}
\frametitle{Debian}
\end{frame}

\begin{frame}
\frametitle{}
\end{frame}

\end{document}
