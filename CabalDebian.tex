%%
%%
%%

\documentclass[cjk,dvipdfm,14pt]{beamer}
\usetheme{Hibi}
\usefonttheme{professionalfonts}
\usepackage{packages}

\title{Cabalの話}
\subtitle{-- Cabalの依存関係解決 --}
%% \subtitle{ -- Debian のすすめ -- }
\author{日比野 啓}
\date{ \today }

\begin{document}

\begin{frame}
\maketitle
\end{frame}

\begin{frame}
\frametitle{自己紹介}

\begin{itemize}
\item {\bf twitter:} @khibino
\item {\bf 仕事:} JavaとかHaskellで\\プログラム書いてます
\item 関数型言語とか\\プログラミング言語処理系とか好きです
\end{itemize}

\end{frame}

%% \begin{frame}

%% \frametitle{Agenda}

%% \begin{itemize}
%% \item Cabal
%% %%\item Cabal
%% \item Debian
%% \end{itemize}

%% \end{frame}

%% \begin{frame}

%% \frametitle{Agenda}

%% \begin{itemize}
%% \item {\color{red} Cabal}
%% %% \item Cabal
%% \item Debian
%% \end{itemize}

%% \end{frame}

\begin{frame}
\frametitle{Haskell のパッケージ}

Hackage \\- Haskell のパッケージを蓄積しているサイト

\begin{itemize}
\item http://hackage.haskell.org/\\packages/archive/pkg-list.html
  \begin{itemize}
  \item パッケージの一覧
  \end{itemize}
\item http://hackage.haskell.org/\\package/{\bf PACKAGE\_NAME}
  \begin{itemize}
  \item 個々のパッケージの情報
  \end{itemize}
\end{itemize}

\end{frame}

\begin{frame}[fragile]
\frametitle{Cabal - install}

%% \lstset{language=bash,basicstyle=\small\ttfamily}
\lstset{language=bash,basicstyle=\ttfamily}

\begin{lstlisting}
% cabal install PACKAGE_NAME
\end{lstlisting}

%% またはソースを展開した場所で

%% \begin{lstlisting}
%% % cabal install
%% \end{lstlisting}

\begin{itemize}
\item 必要となるパッケージが\\全てインストールされる
\item うまくいっているときは便利だが ...
\end{itemize}

\end{frame}

\begin{frame}
\frametitle{Cabalの問題点 - 依存解決 - 1}
%% 問題点その1

\begin{description}
\item ${\bf B}$
\item
  \begin{description}
  \item $C >= 1$ \\{\small … バージョンの制約付き依存関係}
  \item $D >= 1$
  \end{description}
\end{description}

\includegraphics[height=4cm]{./dep10.png}
\end{frame}

\begin{frame}
\frametitle{Cabalの問題点 - 依存解決 - 2}
%% 問題点その1

\begin{description}
\item ${\bf B}$
\item
  \begin{description}
  \item $C >= 1$
  \item $D >= 1$
  \end{description}
\end{description}

\begin{description}
\item ${\bf A}$
\item
  \begin{description}
  \item $B >= 1$
  \item $C >= 1 \wedge C < 2$
  \end{description}
\end{description}

\includegraphics[height=4cm]{./dep11.png}
\end{frame}

\begin{frame}
\frametitle{Cabalの問題点 - 依存解決 - 3}
%% 問題点その1

\begin{description}
\item ${\bf B}$
\item
  \begin{description}
  \item $C >= 1$
  \item $D >= 1$
  \end{description}
\end{description}

\begin{description}
\item ${\bf A}$
\item
  \begin{description}
  \item $B >= 1$
  \item $C >= 1 \wedge {\color{red} C < 2}$
  \end{description}
\end{description}

\includegraphics[height=4cm]{./dep12.png}
\end{frame}

\begin{frame}
\frametitle{Cabalの問題点 - 依存解決 - 4}
%% 問題点その1

\begin{description}
\item ${\bf B}$
\item
  \begin{description}
  \item $C >= 1$
  \item $D >= 1$
  \end{description}
\end{description}

\begin{description}
\item ${\bf A}$
\item
  \begin{description}
  \item $B >= 1$
  \item $C >= 1 \wedge C < 2$
  \end{description}
\end{description}

\includegraphics[height=4cm]{./dep13.png}
\end{frame}

\begin{frame}

\frametitle{Cabalの問題点 - 依存解決 - 5}

\begin{itemize}
\item 必要な試行回数が多くなりすぎる
  \begin{itemize}
  \item 依存関係の段数
  \item パッケージのバージョンの数
  \end{itemize}
\item Cabal はデフォルトでは途中で試行をやめる
\end{itemize}

\end{frame}

\begin{frame}[fragile]
\frametitle{Cabal - 依存解決 - 物量で解決}

\begin{itemize}
\item cabal の試行回数? を明示的に指定
\item デフォルトは 200
\end{itemize}

\begin{lstlisting}
% cabal install [--dry-run] \
    --max-backjumps=1000
\end{lstlisting}

\end{frame}


\begin{frame}
\frametitle{Cabalの問題点 - 壊れる依存関係 - 1}
%% 問題点その2

\begin{description}
\item ${\bf B}$
\item
  \begin{description}
  \item $C >= 1 \wedge C < 2$
  \item $D >= 1$
  \end{description}
\end{description}

\includegraphics[height=4cm]{./dep20.png}

\end{frame}

\begin{frame}
\frametitle{Cabalの問題点 - 壊れる依存関係 - 2}

\begin{description}
\item ${\bf B}$
\item
  \begin{description}
  \item $C >= 1 \wedge {\color{red} C < 2}$
  \item $D >= 1$
  \end{description}
\end{description}

\begin{description}
\item ${\bf A}$
\item
  \begin{description}
  \item $C >= 2$
  \end{description}
\end{description}

\includegraphics[height=4cm]{./dep21.png}

\end{frame}

\begin{frame}[fragile]
\frametitle{Cabalの問題点 - 壊れる依存関係 - 3}

cabal に同時に与えれば、\\両方を満たすように依存関係を解決してくれる

\begin{lstlisting}
% cabal install A-1 B-1 ...
\end{lstlisting}

過去にインストールしたもののうち壊れるものを\\全て与える必要があり大変

\end{frame}


\begin{frame}[fragile]
\frametitle{Cabal - バージョンを指定する}

個別にバージョンを指定しつつ\\インストールすることもできる

\begin{lstlisting}
% cabal unpack A-1.0
% cd A-1.0
% cabal configure
% cabal build
% cabal copy
% cabal register
\end{lstlisting}

依存関係を自前で解決しなくてはならなくて面倒

\end{frame}

\begin{frame}
\frametitle{Debian sid - 1}

\begin{itemize}
\item Debian のパッケージシステムが\\依存関係を壊さないように保ってくれる
\item 517個の hackage (2012-05-27現在) が\\ Debian package 化されている
  \begin{itemize}
  \item yesod や snap もあるよ
  \end{itemize}
\item haskell-platform が更新されない期間も、\\相当するパッケージが提供される
\end{itemize}

\end{frame}

\begin{frame}
\frametitle{Debian sid - 2}

\begin{itemize}
\item Haskell 以外の依存関係も管理されている
  \begin{itemize}
  \item Haskell以外にも依存しているような\\複合的な依存関係でも大丈夫
  \end{itemize}
\item 豊富なパッケージ\\(2012-05-27現在、37526個)
\end{itemize}

\end{frame}

\begin{frame}
\frametitle{まとめ}

\begin{itemize}
\item cabal は便利
\item でも複雑な依存関係を壊さないようにするのは大変
\item Debian sid おすすめです
\end{itemize}

\end{frame}

\begin{frame}
\frametitle{Q\&A}

{\Huge Q\&A}

\end{frame}

\end{document}
